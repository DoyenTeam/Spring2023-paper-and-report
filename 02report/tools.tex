\section{Tool Suite}

Here we describe the tools we as a team will use in executing this project.

\subsection{Communication}
We will use online text and chat tools for our primary means of communication:
\begin{itemize}
    \item Discord: Used for day-to-day communication between team members. We have three channels:
    \begin{itemize}
        \item General: we discuss upcoming deadlines, ask questions still unanswered from meetings, and discuss possible alternatives to current work.
        \item Daily standup: Each member posts a daily update of what they have done the previous day, what they are doing today, and any blockers present. 
        \item Research papers: A channel that houses all of the research that we have found related to any part of the project. 
    \end{itemize}
    We will expand on these as needed over the course of the semester. Discord also allows any-time voice or video calls through the voice channel.
    \item Email: Used to communicate with the client.
\end{itemize}

\subsection{Version Control and CI/CD}
We will use git and GitHub for all of our version control, and we will use GitHub actions for CI/CD. We use a team repository where pull requests will inform the team members of any desired changes to the codebase. This will house front-end code, server code, and any microservices. Upon each accepted commit to the production branch, a continuous integration and delivery pipeline will run the build and appropriate tests using GitHub Actions.

\subsection{Backlog}

We will use GitHub issues to track backlog tasks and issues.

\subsection{File Sharing}

We will use cloud tools for sharing files
\begin{itemize}
    \item Google Drive: Use this to share working files, e.g. meeting notes. 
    \item Overleaf: We use this for working on our core research paper document and the milestones we create from that.
\end{itemize}

\subsection{Documentation}
We will use Swagger for our API documentation and will publish further documentation through GitHub pages, with possible integrations of various auto-doc generators for the different layers of our application to document the code.
