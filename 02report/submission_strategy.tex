\section{Submission Strategy}

There are two key barriers to publication that we will have to overcome. First, we will need to meet the standards and criteria for the journal, and second, we will have to pay for our entry into that journal. Both will be challenging to overcome, given our limited time and funds. However, we found a few journals for which our work may meet the criteria of the journal and which we can consider for publication, depending on the degree of success we find before the end of the semester. We also value open-source publication, so all the journals we consider would publish our work without paywall limitations to access.

First, we considered a journal by the Swiss publisher \href{https://www.mdpi.com/}{MDPI}: the \href{https://www.mdpi.com/journal/ijerph}{International Journal of Environmental Research and Public Health}. Based on the selection of titles listed as recent publications, this journal seems to focus on tools used in, as the name suggests, Environmental Research and Public Health. We believe we have a strong case that our work would be of significant benefit to both communities. Publication in this journal would cost 2,500 CHF, which is roughly 2,755 USD.

Second, we would consider publication with \href{https://journals.plos.org/plosone/s/journal-information}{PLOS ONE}, a journal that will accept any research that is correct without consideration of impact. The journal is viewed by a wide readership. Although we also considered PLOS Biology, it was clear from \href{https://journals.plos.org/plosbiology/s/journal-information#loc-scope}{their scope} that our work would not meet the standards of that journal. PLOS ONE is a journal designed to publish work that does not fit neatly into the boxes of other journals, as can be seen in \href{https://journals.plos.org/plosone/s/journal-information#loc-scope}{their scope}. The fees for publishing in PLOS one could be as low as 856 USD or as high as 1,931 USD, which would still be considerably less than MDPI. We believe this is a realistic option for the publication of our work.

Last but not least, we would most certainly have the ability to publish in \href{https://joss.theoj.org/}{Journal of Open Source Software (JOSS)}, which is an open-source, peer-reviewed, free-to-publish online journal hosted using GitHub. As the name suggests, they will publish any open-source software that could make an impact on "the functioning of research instruments or the execution of research experiments". They use a modern open-science method of reviews that focuses on getting submissions up to the standards rather than weeding out submissions. If all else fails, if we are able to complete our work even remotely well, we should be able to publish with JOSS.

Regardless of what journal we publish, the topic of our project has illuminated the fact that our work will be made visible to biologists through PubMed and, as these are all open-access journals, likely PMC Open Access. For example, the work published by a good friend and colleague of one of our team members -- whose experience was consulted when considering our options for publications -- was \href{https://joss.theoj.org/papers/10.21105/joss.01708}{published in JOSS}, then \href{https://pubmed.ncbi.nlm.nih.gov/32337477/}{indexed by pubmed}, and is now also \href{https://joss.theoj.org/papers/10.21105/joss.01708}{available on PMC OA}.